\documentclass[a4paper]{article}
    \usepackage{hyperref}
    \usepackage{fullpage}
    \usepackage{amsmath}
    \usepackage{amssymb}
    \usepackage{textcomp}
    \usepackage[utf8]{inputenc}
    \usepackage[T1]{fontenc}
    \textheight=5in
    \pagestyle{empty}
    \raggedright
    \usepackage[left=0.4in,right=0.4in,bottom=0.2in,top=0.5in]{geometry}
\usepackage{etoolbox,refcount}
\usepackage{multicol}


\newcounter{countitems}
\newcounter{nextitemizecount}
\newcommand{\setupcountitems}{%
  \stepcounter{nextitemizecount}%
  \setcounter{countitems}{0}%
  \preto\item{\stepcounter{countitems}}%
}
\makeatletter
\newcommand{\computecountitems}{%
  \edef\@currentlabel{\number\c@countitems}%
  \label{countitems@\number\numexpr\value{nextitemizecount}-1\relax}%
}
\newcommand{\nextitemizecount}{%
  \getrefnumber{countitems@\number\c@nextitemizecount}%
}
\newcommand{\previtemizecount}{%
  \getrefnumber{countitems@\number\numexpr\value{nextitemizecount}-1\relax}%
}
\makeatother    
\newenvironment{AutoMultiColItemize}{%
\ifnumcomp{\nextitemizecount}{>}{3}{\begin{multicols}{2}}{}%
\setupcountitems\begin{itemize}}%
{\end{itemize}%
\unskip\computecountitems\ifnumcomp{\previtemizecount}{>}{3}{\end{multicols}}{}}


    %\renewcommand{\encodingdefault}{cg}
%\renewcommand{\rmdefault}{lgrcmr}

\def\bull{\vrule height 0.7ex width .7ex depth -.1ex }

% DEFINITIONS FOR RESUME %%%%%%%%%%%%%%%%%%%%%%%
\hypersetup{
    colorlinks=true,
    linkcolor=black,
    filecolor=magenta,      
    urlcolor=black,
    pdftitle={Abhishek's Resume},
    pdfpagemode=FullScreen,
    }

\newcommand{\area} [2] {
    \vspace*{-9pt}
    \begin{verse}
        \textbf{#1}   #2 
    \end{verse}
}

\newcommand{\lineunder} {
    \vspace*{-8pt} \\
    \hspace*{-18pt} \hrulefill \\
}

\newcommand{\header} [1] {
    {\hspace*{-18pt}\vspace*{6pt} \textsc{#1}}
    \vspace*{-6pt} \lineunder
}

\newcommand{\employer} [3] {
    { \textbf{#1} (#2)\\ \underline{\textbf{\emph{#3}}}\\  }
}

\newcommand{\contact} [3] {
    \vspace*{-10pt}
    \begin{center}
        {\Huge \scshape {#1}}\\
        #2 \\ #3
    \end{center}
    \vspace*{-8pt}
}

\newenvironment{achievements}{
    \begin{list}
        {$\bullet$}{\topsep 0pt \itemsep -2pt}}{\vspace*{4pt}
    \end{list}
}

\newcommand{\schoolwithcourses} [4] {
    \textbf{#1} #2 $\bullet$ #3\\
    #4 \\
    \vspace*{5pt}
}

\newcommand{\school} [4] {
    \textbf{#1} #2 $\bullet$ #3\\
    #4 \\
}
% END RESUME DEFINITIONS %%%%%%%%%%%%%%%%%%%%%%%

    \begin{document}
\vspace*{-40pt}

    

%==== Profile ====%
\vspace*{-9pt}
\begin{center}
	{\Huge \scshape {Abhishek Sokhal}}\\
	\vspace{2mm}
	India $\cdot$ abhisheksokhal2035@gmail.com $\cdot$ +91 9769388293 $\cdot$ \href{https://github.com/Distroto}{GitHub} $\cdot$ \href{https://www.linkedin.com/in/abhishek-sokhal-630a951ba/}{Linkedin} \\
\end{center}
\vspace{-5mm}
\begin{center}
Dedicated, goal-oriented, and team-oriented individual who seeks an opportunity to work in the field of Game Programming and Development\\
\end{center}
\vspace{-1mm}
%==== Education ====%
\header{Education}
\vspace{0mm}
\textbf{Indian Institute of Information Technology}\hfill Bhopal\\
BTech Information Technology \hfill December 2021 - Present\\
\vspace{-1mm}

%==== Work Experience ====%
\header{Work Experience}
\vspace{0mm}
\textbf{Shemaroo Entertainment Ltd}\hfill Remote\\
{\textit {Unity Developer Intern}} \hfill February 2022 - Present\\


\hspace{1.5em}\textbf{ShemarooVerse - Jio Dive} 
\begin{itemize} 
\vspace{-2.5mm}
        \item Conceptualized and designing a VR metaverse using JioMixedReality Toolkit for Jio Dive. 
 \vspace{-2.5mm}
        \item Integrated an OTT API using Unity's networking and HTTP libraries, enabling in-app multimedia streaming.
 \vspace{-2.5mm} 
        \item Added user teleportation to multiple positions within a specific scene and facilitating interactive engagement with objects, while seamlessly streaming content in other scenes.
 \vspace{-2.5mm} 
\end{itemize}

\hspace{1.5em}\textbf{ShemarooVerse - Android} 
\begin{itemize} 
\vspace{-2.5mm}
        \item Leading the development of a metaverse application using Unity, focusing on scalability and optimal user experience.
 \vspace{-2.5mm}
       \item Engineered immersive 3D avatars  and dynamic animations utilizing Unity's suite of tools and 3rd party plugins.
 \vspace{-2.5mm}
       \item Working with multiplayer SDK(Photon), adeptly implementing various features.
 \end{itemize}
\vspace{1mm}

\textbf{Irusu Private Ltd.}\hfill Remote\\
{\textit {Unity Developer Intern}} \hfill September 2022 - January 2023 \\ 

\hspace{1.5em}\textbf{Irusu Block} 
\begin{itemize} 
\vspace{-2.5mm}
    \item This AR app utilizes ARCore SDK to detect surfaces in the real world and superimpose 3D objects onto them using the phone's camera.
 \vspace{-2.5mm}
    \item We are able to break down the 3D objects into different components and each component can be resized and moved around in the 3D plane.
 \vspace{-2.5mm}
\end{itemize}

\hspace{1em}\textbf{Irusu Metaverse} 
\begin{itemize} 
\vspace{-2.5mm}
    \item This app utilizes Unity Addressables to load various scenes dynamically while the app is running. 
 \vspace{-2.5mm}
    \item Additionally, a lobby system is implemented with Agora SDK, enabling users to engage in voice chats with each other.
 \vspace{-2.5mm}
\end{itemize}

\vspace{1mm}
%==== Position of Responsibility ====%
\header{Position of Responsibility}
\vspace{0mm}
\textbf{GNU/Linux Users Club}\hfill August 2022 - Present\\
{\textit {Core Team Member}}  
\vspace{-1mm}
\begin{itemize} 
	\item Using Arch-based distributions as my secondary operating system but have also have experienced Fedora and most Debian Based distributions.
     \vspace{-2mm}
        \item Guiding batch-mates to install and use Linux distribution comfortably as well as troubleshooting problems for them. 
\end{itemize}

\vspace{1mm}

%==== Projects ====%
\header{Projects}
{\textbf{Multiplayer Game}} \hfill November 2022
\vspace{-2.5mm}
\begin{itemize} 
	\item Made in Unity 3D using Netcode for game objects
  \vspace{-2.5mm}
        \item Used Unity Gaming Services to create and manage lobbies 
\end{itemize}
{\textbf{API Request App}} \hfill September 2022
\vspace{-2.5mm}
\begin{itemize} 
	\item App made in Unity for android devices
  \vspace{-2.5mm}
        \item Fetches data from a JSON file and displays it in a scrollable list
\end{itemize}
{\textbf{VR Color Changing App}} \hfill August 2022
\vspace{-2.5mm}
\begin{itemize} 
	\item Made in Unity using Google VR SDK
  \vspace{-2.5mm}
        \item Basic App which will changes color of the objects in the environment once the pointer is placed on it and after it leaves
\end{itemize}
{\textbf{Pong}} \hfill July 2022
\vspace{-2.5mm}
\begin{itemize} 
	\item Developed a replica of the retro hit pong in Unity 2D and C-Sharp scripts
  \vspace{-2.5mm}
        \item Automated the player2 to make it a player vs computer game 
\end{itemize}
%==== Skills ====%
\header{Skills}
\vspace{2mm}
\begin{tabular}{ l l }
	Programming Languages: & C++ ,C, C-Sharp, Lua, Python \\
	Game Engines:& Unity, Love 2D, Unreal, UEFN(Fortninte Builder)  \\
        Devops Tools:        & Latex, GitHub, Git, GitLFS, Azure \\  
	Web Development:       & HTML , CSS (Bootstrap and Tailwind), JavaScript   \\
	Soft Skills:           &  Leadership, Communication Skills, Organised  \\
\end{tabular}

%==== Achievements ====%
\vspace{2mm}
% \header{}
\header{Achievements}
\textbf{Runner Up} \hfill GDSC IIIT Bhopal\\
Won second prize in the Intra IIIT development contest conducted by GDSC IIIT Bhopal for creating a password manager.  
{\textit {Skills Used: Python, MySQl, Encryption Methods (MD5Sum)}}  \hfill 2022\\
\end{document}